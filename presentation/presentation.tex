\documentclass{beamer}
\usepackage{graphicx}
\usetheme{default}
\usecolortheme{beaver}
\let\oldAA\AA
\let\oldaa\aa
\def\AA{\^ A}
\def\aa{\^ a}
\def\myqq@{'}
\catcode`\'=13
\def'#1{\ifmmode {}^\prime #1%
         \else\ifx#1a\u{a}%
         \else\ifx#1i\^{\i}%
         \else\ifx#1s\c{s}%
         \else\ifx#1t\c{t}%
         \else\ifx#1A\u{A}%
         \else\ifx#1I\^{I}%
         \else\ifx#1S\c{S}%
         \else\ifx#1T\c{T}%
         \else\ifx#1"\myqq@%
               \fi\fi\fi\fi\fi\fi\fi\fi\fi%
         \fi}

\usepackage{listings}         
\lstset{breakatwhitespace,
tabsize=3,
language=Java,
columns=fullflexible,
keepspaces,
breaklines,
tabsize=3, 
showstringspaces=false,
extendedchars=true}
\newcommand{\TOSAY}[1]{\textcolor{green}{TOSAY: #1}}
\newcommand{\TODO}[1]{\textcolor{red}{TODO: #1}}


\title[]{Side-effect disparity as a metric for identifying behavioral subtyping violations}
\author{Lor\'{a}nd Szak\'{a}cs}
\institute{Iowa State University}
\date{December 9 2013}


\begin{document}

\begin{frame}
\titlepage
\end{frame}

\Large

\section{Introduction}
\begin{frame}{Motivation}
\huge
\begin{itemize}
  \item  Behavioral sub-typing is well known heuristic in OOP
 \end{itemize}
 \tiny
R. C. Martin. Agile software development: principles, patterns, and practices. Prentice Hall PTR, 2003
 \huge
 \begin{itemize} 
  \item \pause Currently, formally defined in terms of specification
\end{itemize}
\end{frame}

\begin{frame}{Motivation}
\huge
Disadvantages of formal specification:
\begin{itemize}
  \item \pause someone has to write it
  \item \pause something has to read it
\end{itemize}
\end{frame}

\begin{frame}{Behavioral sub-typing}
\huge
\textbf{An alternative}
\begin{itemize}
  \item \pause observe behavior through side-effects
  \item \pause specifically, the difference in side-effects
  \item \pause let empirical observation lead the way
\end{itemize}
\end{frame}

\begin{frame}{Motivation}
\Huge
The previous research question:
\huge
\begin{itemize}
  \item \pause Does behavioral sub-typing happen in practice?
\end{itemize}
\end{frame} 

\begin{frame}{A Simple Heuristic}
\Huge
report a violation if there is a difference between the side-effects of a method in the subclass and superclass
\end{frame}

\begin{frame}{Side Effects}
\Huge
What is a side-effect?
\huge
\begin{itemize}
  \item \pause assignment to a field
  \item \pause method call on a field
\end{itemize}
\end{frame}

\begin{frame}{Boa}
\huge
\textbf{Boa}
\begin{itemize}
  \item \pause language and framework for large scale software
\end{itemize}

\end{frame}

\begin{frame}{Contribution}
\huge
\begin{itemize}
  \item First exploration of the use of side-effects for identifying behavioral sub-typing violations
\end{itemize}
\end{frame}

\section{Empirical evaluation}
\begin{frame}{Looking at real world code}
\Huge
\center{5113 Java projects}
\end{frame}

\begin{frame}{Looking at real world code}
\Huge
\center{1286 met the preconditions}
\normalsize


\pause \begin{itemize}
\item contain dirty kind of sub-typing \pause i.e. override non-abstract methods
\end{itemize}
\Large

\pause A nice side-effect of this project, \pause hehe

\normalsize
\pause \begin{itemize}
\item extend only classes accessible from within the project
\end{itemize}
\end{frame}

\begin{frame}{Looking at Real World Code}
\Huge
\center{739 projects contained at least a violation}
\end{frame}

\begin{frame}{Looking at Real World Code}
\Huge
\center{101,433 pairs of subtype-supertype}
\end{frame}

\begin{frame}{Looking at Real World Code}
\Huge
\center{11,600 were violations}
\end{frame}

\begin{frame}{Looking at Real World Code}
\Huge
\center{2,701,596 overridden methods}
\end{frame}

\begin{frame}{Looking at Real World Code}
\Huge
\center{283,216 were violations}
\end{frame}

%\begin{frame}[fragile]{Loop carried dependency}
%\begin{lstlisting}
%//called from within the loop 
%public void parallelContext() {
%    //notice how the field "LCD" is first read
%    int temp = this.LCD;
%    //...
%    this.LCD = 42;
%    //...
%}
%\end{lstlisting}
%\end{frame}

\begin{frame}{Evaluation}
\Huge
\center{Manually inspected 20 violations}


\Large
\begin{itemize}
  \item \pause 1 false positive due to incorrect implementation
  \item \pause 2 false positives due to subtype method and supertype  method writing to completely different fields.
  \item \pause 6 false positives due to template pattern
\end{itemize}
\end{frame}

\begin{frame}[fragile]{Template pattern}
\begin{lstlisting}
//super code:
public void validate() throws XPathException {}

//subclass code:
public void validate() throws XPathException {
   checkWithinTemplate();
   test = typeCheck("test", test);
}
\end{lstlisting}
\end{frame}

\begin{frame}{How it works}
\Huge
  Look away...
\end{frame}

\begin{frame}{Threats to validity}
\huge
\begin{itemize} 
  \item no independent means of verifying behavioral sub-typing
  \item \pause no idea how to pigeonhole the numbers 
  \item \pause although it did not pose a problem in the small random sample\pause, the broad definition of side-effects might cause problems
\end{itemize}
\end{frame}

\begin{frame}{Threats to validity}
\huge
\begin{itemize} 
  \item behavioral subtyping concerns itself with specific values
\end{itemize}
\end{frame}

\begin{frame}[fragile]{Room for improvement}
\huge
\begin{itemize}
\item improve heuristic
\item \pause make the identification of a side-effect more precise
\item \pause account for actual side-effects of method calls
\item \pause consider side-effects on parameters
\item \pause more data

\end{itemize}
\end{frame}



\begin{frame}{In Summary}
\Large
\begin{itemize}
\item We defined a simple heuristic for identifying behavioral subtyping violations.
\item \pause evaluated said heuristic on 5113 projects
\item \pause 11,600 out of 101,433 pairs of subtype-supertype contained violations
\item \pause 283,216 out of 2,701,596 overridden methods violated behavioral subtyping

\end{itemize}
\end{frame}

\begin{frame}{Questions}
  \Huge
   \center{Questions?}
\end{frame}


\end{document}
